\documentclass[12 pt, oneside]{book}
\usepackage[english]{babel}
\usepackage[utf8x]{inputenc}
\usepackage{amsmath}
\usepackage{amssymb}
\usepackage{changepage}
\usepackage[right, displaymath]{lineno}




\begin{document}

\newenvironment{mathline}{\begin{adjustwidth}{2cm}{}\begin{linenumbers*}}{\end{linenumbers*}\end{adjustwidth}}
\newcommand{\laplace}{$\mathcal{L}$}

\frontmatter

\title{AMS361 - Differential Equations}
\author{Lecturer: Dr. Yuefan Deng}
\date{Spring 2017 Term}

\maketitle

\tableofcontents

\chapter{Introduction}

\subsection{Lecture Information}
Class: AMS 361: Differential Equations\newline
Lecturer: Dr. Yuefan Deng\newline
Term: Spring 2017\newline
Location: Earth and Space Sciences 101\newline
Time: Tuesday, Thursday 5:30 p.m.-6:50 p.m.\newline


\mainmatter

\chapter{Introduction to Linear Equations}
\section{Characteristics of Differential Equations}
\begin{itemize}
\item \underline{linear vs nonlinear} - an equation is linear if the dependent variable or its derivative is linear
\item \underline{order of ordinary differential equations} - the order of an ordinary differential equation is the number of the highest derivative in the equation
\item \underline{homogeneous vs inhomogeneous} - an equation is homogeneous if it can be arranged so its dependent variable terms are equal to 0
\end{itemize}

\begin{tabular}{|p{.15\linewidth}|p{.15\linewidth}|p{.35\linewidth}|p{.35\linewidth}|}
  \hline
  num. DV & num. IV & 1 & 2+\\
  \hline
  1 & &ordinary differential equation & partial differential equation\\
  \hline
  2 & &system of oridnary differential equations & system of partial differential equations\\
  \hline
\end{tabular}

\chapter{First Order Equations}
\section{Separable First Order Equations}
Form:
\begin{mathline}
$y'=\frac{f(x)}{g(y)}$
\end{mathline}
Solution:
\begin{mathline}
$\frac{dy}{dx}=\frac{f(x)}{g(y)}$\newline{}
$g(y)dy=f(x)dx$\newline{}
$\int g(y)dy = \int f(x)dx$\newline{}
integrate and solve for $y$  
\end{mathline}

\section{Linear First Order Equations}
Form:
\begin{mathline}
$y'+P(x)y+Q(x)=0$
\end{mathline}
Solution Derivation:
\begin{mathline}
$y'+P(x)y+Q(x)=0$\newline{}
$v(x)[y'+P(x)y+Q(x)=0]$ - multiply by some factor $v(x)$\newline{}
$v(x)y'+v(x)P(x)y+v(x)Q(x)=0$\newline{}
  \indent let $v'(x)=v(x)P(x)$ - this way,  $\frac{d[v(x)y(x)]}{dx}=v(x)y'+v(x)P(x)y$\newline{}
  \indent $\frac{dv}{dx}=v(x)P(x)$\newline{}
  \indent $\int \frac{dv}{v(x)}=\int P(x)dx$\newline{} 
  \indent $ln(v(x))=\int P(x)dx + c$ \newline{}
  \indent $v(x)=e^{\int P(x)dx + c}=Ae^{\int P(x)dx}$\newline{}
$v(x)y'+v(x)P(x)y+v(x)Q(x)=0$\newline{}
plug in $v(x)$ and solve as a first order linear differential equation
\end{mathline}
Solution:
\begin{mathline}
$y'+P(x)y+Q(x)=0$\newline{}
let $v(x)=Ae^{\int P(x)dx}$\newline{}
$v(x)[y'+P(x)y+Q(x)=0]$\newline{}
$v(x)y'+v(x)P(x)y+v(x)Q(x)=0$\newline{}
substitute and solve as a first order linear differential equation
\end{mathline}

\section{Polynomial Functions}
Form:
\begin{mathline}
$y'=F(ax+by+c)$
\end{mathline}
Solution Derivation:
\begin{mathline}
$y'=F(ax+by+c)$\newline{}
\indent let $u=ax+by+c$\newline{}
\indent$u'=a+by'$\newline{}
\indent$\frac{dy}{dx}=(\frac{du}{dx}-a)\frac{1}{b}$ - solve for y'\newline{}
$y'=F(u)=\frac{u'-a}{b}$\newline{}
\end{mathline}
Solution:
\begin{mathline}
$y'=F(ax+by+c)$\newline{}
let $u=ax+by+c$\newline{}
$y'=F(u)=\frac{u'-a}{b}$\newline{}
solve for $u$ as a separable first order linear differential equation\newline{}
substitute and solve for $y$
\end{mathline}

\section{Homogeneous Differential Equations}
Form:
\begin{mathline}
$y'=F(\frac{y}{x})$
\end{mathline}
Solution Derivation:
\begin{mathline}
$y'=F(\frac{y}{x})$\newline{}
\indent let $u=\frac{y}{x}$\newline{}
\indent $ux=y$\newline{}
\indent $u'x+u=y'$\newline{}
$y'=F(\frac{y}{x})=F(u)=u'x+u$\newline{}
$F(u)-u=\frac{du}{dx}x$\newline{}
$\frac{dx}{x}=\frac{du}{F(u)-u}$\newline{}
$\int \frac{dx}{x}=\int \frac{du}{F(u)-u}$\newline{}
$ln(x)=\int \frac{du}{F(u)-u}+c$\newline{}
$x=Ae^{\int \frac{du}{F(u)-u}}$\newline{}
solve for $u$, then substitute and solve back for $y$
\end{mathline}
Solution:
\begin{mathline}
$y'=F(\frac{y}{x})$\newline{}
let $u=\frac{y}{x}$\newline{}
$x=Ae^{\int \frac{du}{F(u)-u}}$\newline{}
solve for $u$, then substitute and solve back for $y$
\end{mathline}

\section{Bernoulli Differential Equations}
Form:
\begin{mathline}
$y'+P(x)y=Q(x)y^n$
\end{mathline}
Solution:
\begin{mathline}
$y'+P(x)y=Q(x)y^n$\newline{}
$y'y^{-n}+P(x)y^{1-n}=Q(x)$\newline{}
\indent let $u=y^{1-n}$\newline{}
\indent $u'=(1-n)y^{-n}y'$\newline{}
$\frac{1}{1-n}u'+P(x)u=Q(x)$
solve for $u$ as a separable first order linear equation
\end{mathline}

\section{Ricatti Differential Equation}
Form:
\begin{mathline}
$y'=A_0(x)+A_1(x)y+A_2(x)y^2$, given $y_0(x)$
\end{mathline}
Solution:
\begin{mathline}
$y_0(x)=A_0(x)+A_1(y)+A_2(x)y^2$, given $y_0(x)$\newline{}
$y_0'=A_0(x)+A_1(x)y_0+A_2(x)y_0^2$\newline{}
\indent the general solution will be of the form $y=y_0(x)+\frac{1}{z(x)}$\newline{}
\indent $y^2=y_0(x)^2+2y_0(x)\frac{1}{z(x)}+\frac{1}{z(x)^2}$\newline{}
\indent $y'=y_0'(x)-\frac{1}{z(x)^2}z(x)'$\newline{}
\indent $y'=y_0'(x)-\frac{1}{z^2}z'=A_0(x)+A_1(x)(y_0+\frac{1}{z})+A_2(x)(y_0^2+2y_0\frac{1}{z}+\frac{1}{z^2}$\newline{}
\indent $y_0'=A_0(x)+A_1(x)y_0+A_2(x)y_0^2$ - subtract
\indent $-\frac{1}{z^2}z'=A_1(x)\frac{1}{z}+A_2(x)\frac{2y_0}{z}+\frac{A_2(x)}{z^2}$\newline{}
\indent $z'=-A_1(x)z(x)-A_2(x)2y_0(x)z(x)-A_2(x)$\newline{}
\indent $z'=(-A_1-2A_2y_0)z(x)-A_2(x)$\newline{}
solve as a first order linear differential equation
\end{mathline}

\section{Exact Differential Equations}
Form:
\begin{mathline}
$M(x,y)dx + N(x,y)dy = 0$\newline{}
  the equation is exact if $\frac{M(x,y)}{\delta y}=\frac{N(x,y)}{\delta x}$
\end{mathline}
Solution:\newline{}
\indent if the equation is not exact, multiply it by a factor $\rho$
\begin{mathline}
  $\rho[M(x,y)dx + N(x,y)dy = 0]$\newline{}
  $\rho(x) M(x,y) = \rho(x) N(x,y)$ or $\rho(y) M(x,y) = \rho(y) N(x,y)$\newline{}  
  $\frac{d (\rho M(x,y))}{dy} = \frac{d(\rho N(x,y))}{dx}$\newline{}
  $\rho(x) M_y = \rho'(x)N +\rho(x)N_x$ or $\rho'(y)M + \rho(y)M_y = \rho(y) N_x$\newline{}
  $\rho(x)[M_y-N_x]=\rho'(x)N$ or $\rho(y)[M_y-N_x]=-\rho'(y)M$\newline{}
  $\rho(x)\frac{M_y-N_x}{N}=\frac{d\rho}{dx}$ or $\rho(y)\frac{M_y-N_x}{M}=-\frac{d\rho}{dy}$\newline{}
  $\int \frac{M_y-N_x}{N} dx = \int \frac{d\rho}{\rho(x)}$ or $\int \frac{M_y-N_x}{M} = \int -\frac{d\rho(y)}{dy}$\newline{}
  $\rho=e^{\int\frac{M_y-N_x}{N}dx}$ or $\rho=e^{\int\frac{N_x-M_y}{M}dy}$
\end{mathline}
\indent \indent if the equation is exact:
\begin{mathline}
  $F_1=\int M(x,y)dx$\newline{}
  $F_1(x,y)+g(y)$\newline{}
  $F_2=\int N(x,y) dy$\newline{}
  $F_2(x,y)+h(x)$\newline{}
  combine $F_1$ and $F_2$ to yield $f(x)$
\end{mathline}

\chapter{Mathematical Models}

\section{Newton's Equation of Cooling}

Newton's Law of Cooling formula is used to calculate the temperature of an object as it loses heat as a function of its original body temperature, the environmental temperature, and time.

Form:
\begin{mathline}
  $\frac{dT}{dt}=k(A-T)$\newline{}
  $T(t=0)=T_0$
\end{mathline}

Variables:
\begin{mathline}
  $T$ = temperature of the body\newline{}
  $A$ = temperature of the environment\newline{}
  $k$ = thermal conductivity\newline{}
  $t$ = time
\end{mathline}

\section{Toricelli's Draining Equation}
Toricelli's law relates the speed of fluid exiting from an opening in a container with the height of the fluid above the opening.

Form:
\begin{mathline}
  $\frac{dV}{dt}=\frac{A(y)dy}{dt}=-k\sqrt{g}$\newline{}
  $y(t=0)=y_0$
\end{mathline}

Variables:
\begin{mathline}
$V$ = volume of fluid in the container\newline{}
$A$ = area of the upper surface of the fluid in the container\newline{} 
$y$ = height of fluid over the opening\newline{}
$k$ = constant based on viscosity of water, other variables\newline{}
$g$ = constant of accelereation due to gravity
\end{mathline}

\section{Population Differential Equation}
This is the logistic model of the population growth model equation.

Form:
\begin{mathline}
  $\frac{dP}{dt}=kP(M-P)=kMP-kP^2$\newline{}
  $P(t=0)=P_0$
\end{mathline}

Variables:
\begin{mathline}
  $P$ = population\newline{}
  $M$ = carrying capacity\newline{}
  $k$ = rate of growth\newline{}
  $t$ = time
\end{mathline}

\section{Rocket Differential Equation}
This model is used to find the velocity of a rocket fired in into the atmosphere, taking into account air resistance. This equation can be modified to suit any scenario that involves applying external forces to a moving mass.

Form:
\begin{mathline}
  $m\frac{dv}{dt}=-mg-kv$\newline{}
  $v(t=0)=v_0$
\end{mathline}

Variables:
\begin{mathline}
  $v$ = velocity of the rocket \newline{}
  $m$ = mass of the rocket\newline{}
  $g$ = acceleration due to gravity\newline{}
  $k$ = drag/air resistant coefficient\newline{}
  $t$ = time
\end{mathline}
  
\section{Finance Differential Equation}
This equation is used to model interest.

Form:
\begin{mathline}
  $\frac{dz}{dt} = zr-w$\newline{}
  $z(t=0)=z_0$
\end{mathline}

Variables:
\begin{mathline}
  $z$ = initial investment\newline{}
  $r$ = interest rate\newline{}
  $w$ = payments\newline{}
  $t$ = time
\end{mathline}

\section{Swimmer's Problem}
This equation is used to model a swimmer trying to cross a river with a current.

Form:
\begin{mathline}
  $v_w(x)=\frac{dy}{dx}=\frac{v_c}{v_s}(1-(\frac{x}{a})^2)$
\end{mathline}

Variables:
\begin{mathline}
  $v_c$ = velocity of the current (perpendicular to swimmer)\newline{}
  $v_s$ = velocity of the swimmer (perpendicular to current)\newline{}
  $y$ = distance from the shortest path across from the start position(parallel to current)\newline{}
  $x$ = position along shortest path between shores (perpendicular to current)\newline{}
  $a$ = half the distance of the distance between shores
\end{mathline}

\section{Plane Landing Equation}
This equation is used model the landing of a plane through perpendicular winds, assuming the plane continually adjusts its angle to face the destination

Form:
\begin{mathline}
  $y=\frac{x}{2}[(\frac{x}{a})^{-\alpha}-(\frac{x}{a})^{\alpha}]$\newline{}
  $\frac{dy}{dx}=\frac{y}{x}-\frac{W}{v_0}\sqrt{1+\frac{y^2}{x^2}}$\newline{}
  $\frac{y}{x}+\sqrt{1+\frac{y^2}{x^2}}=(\frac{x}{a})^{-\alpha}$\newline{}
  $\alpha = \frac{W}{v_0}$
\end{mathline}

Variables:
\begin{mathline}
  $a$ = distance at which descent begins\newline{}
  $w$ = wind (perpendicular to plane's route)\newline{}
  $v_0$ = plane's velocity
\end{mathline}
  
\chapter{Higher Order Differential Equations}

\section{General Overview of Higher Order Differentials}
Higher order differential equations of the second degree follow the form:\newline{}

\begin{mathline}
  $a_2(x)y''+a_1(x)y'+a_0(x)y=f(x)$
\end{mathline}

\begin{tabular}{|p{.1\linewidth}|p{.1\linewidth}|p{.4\linewidth}|p{.4\linewidth}|}
  \hline
  $f(x)$ & $a_i(x)$ & constant & variable\\
  \hline
  $0$ & & characteristic equation method & cauchy-euler \\
  \hline
  $\neq 0$ & & method of undetermined coefficients or variation of parameters & variation of parameters\\
  \hline
\end{tabular}

\section{Homogeneous Linear Differential Equations}
This is the characteristic equation method, where one side consisting of all dependent variable terms and their derivatives is converted to a simple polynomial equation in terms of $\lambda$

Form:\newline{}
\begin{mathline}
$a_ny^{(n)}+a_{n-1}y^{(n-1)}+...+a_1y^{(1)}+a_0y=0$\newline{}
make substitution $y=e^{\lambda x}$ -> this is the only way a derivative will not go to zero due to falling order or produce a coefficient with $x$\newline{}
$a_{n}\lambda^{n}e^{\lambda x}+a_{n-1}\lambda^{n-1}e^{\lambda x}+...+a_{1}\lambda e^{\lambda x}+a_0+e^{\lambda x}=0$\newline{}
$e^{\lambda x}(a_{n}\lambda^{n}+a_{n-1}\lambda^{n-1}+...+a_{1}\lambda +a_0)=0$\newline{}
$e^{\lambda x}\neq 0$\newline{}
$a_{n}\lambda^{n}+a_{n-1}\lambda^{n-1}+...+a_{1}\lambda+a_0=0$\newline{}
solve for $\lambda$ and substitute to solve for $y$\newline{}
$y_{GS}=c_0y_0+c_1y_1+...+c_{n-1}y_{n-1}+c_{n}y_n$
\end{mathline}
\subsection{Addressing Duplicates}
When solving the characteristic equation, there is a possibility that there can be duplicate values for $\lambda$ and $y$. In the case that there is more than one solution of $y_i=e^{cx}$, multiply duplicate solutions by the independent variable, in this case $x$, to produce a solution that is linearly independent from the others. The reasoning behind this is that $\frac{d(xe^{cx})}{dx}=e^{cx}+xe^cx$, so the extra terms get absorbed into lower derivative values.

\section{Euler-Cauchy Linear Differential Equations}
Form:\newline{}
\begin{mathline}
$x^{n}y^{(n)}+x^{n-1}y^{(n-1)}+...+xy^(1)+y=0$\newline{}
\indent let $y=x^\lambda$ -> when multiplied with the coefficients, will produce terms all of the same order\newline{}
$x^{n}(x^\lambda)^{(n)}+x^{n-1}(x^\lambda)^{(n-1)}+...+(x^\lambda)y^(1)+(x^\lambda)=0$\newline{}
$x^{n}[(\lambda)(\lambda-1)...(\lambda-n)]x^{(\lambda-n)}+...+x\lambda x^{\lambda-1}+x^\lambda=0$\newline{}
$x^{n}[(\lambda)(\lambda-1)...(\lambda-n)+...+\lambda+1]=0$\newline{}
$x^\lambda\neq 0$\newline{}
$(\lambda)(\lambda-1)...(\lambda-n)+...+\lambda+1=0$\newline{}
solve for $\lambda$ and substitute to solve for $y$
\end{mathline}

\subsection{Addressing Duplicates}
In the case of duplicate solutions for $y_i=x^\lambda$, multiply solutions by $ln(x)$ until all solutions are of the form $y_i=ln(x)^{c}x^\lambda$ and linearly independent.

\section{Inhomogeneous Differential Equations}

\subsection{Finding a Particular Solution}

Theorem: The general solution to an inhomogeneous differential equation is the sum of the characteristic equation of the dependant terms and a particular solution to the equation.

$y_{GS}(x)=y_c(x)+y_p(x)$

For any extra terms not included in the characteristic equation, one can find a trial solution by setting a $y$ value to all extra terms with variable coefficients. Polynomial terms must be reduced as well.

\subsection{Reduction of Order}

Given an higher order linear differential equation, $a_2xy''+a_1xy'+a_0xy=0$ and a particular solution $y_1(x)$, we can get a second solution.

$y=u(x)y_1$\newline{}
$y'=u(x)y_1'+u'(x)y_1$\newline{}
$y''=u(x)y_1''+2u'(x)y_1'+u''(x)y_1$\newline{}
$ln(u)=-\int a_1+2\frac{y_1'}{y+1} dx + c_0$\newline{}
$u(x)=c_1\int e^{-\int a_1+2\frac{y_1'}{y+1} dx}+c_2$\newline{}
$y=u(x)y_1(x)=y_1[c_1\int e^{-\int a_1+2\frac{y_1'}{y+1} dx}+c_2]$

\subsection{Variational Principle Method}

Given an equation $y''+P(x)y'+Q(x)y=f(x)$

The solution of a higher order differential equation requires two solutions, $y_1(x)$, $y_2(x)$

Propose a particular solution:

\begin{mathline}
  $y_p(x)=u_1y_1+u_2y_2$
  $y_p'(x)=u_1'y_1+u_1y_1'+u_2'y_2+u_2y_2'$
  \indent assume $u_1'y_1+u_2'y_2=0$
  $y_p'(x)=u_1y_1'+u_2y_2'$
  $y_p''(x)=u_1'y_1'+u_1y_1''+u_2'y_2'+u_2y_2''$
  plug into the original equation
  $ u_1'y_1'+u_1y_1''+u_2'y_2'+u_2y_2'' + P(x)(u_1y_1'+u_2y_2') + Q(x)(u_1y_1+u_2y_2)=f(x)$
  $(u_1'y_1'+u_2'y_2')+u_1(y_1''+P(x)y_1'+Q(x)y_1)+u_2(y_2''+P(x)y_2'+Q(x)y_2)=f(x)$
  $(u_1'y_1'+u_2'y_2')+u_1x(0)+u_2x(0)=f(x)$
  $u_1'y_1'+u_2'y_2'=f(x)$
  $u_1'y_1+u_2'y_2=0$ was an earlier assumption
  $u_1'=\frac{y_2f(x)}{y_1y_2'-y_2y_1'}$ - solve the system for $u_1'$
  $u_2'=\frac{y_1f(x)}{y_1y_2'-y_2y_1'}$ - solve the system for $u_2'$
  let $W(y_1,y_2)=y_1y_2'-y_2y_1'\neq 0$
  $u_1=-\int\frac{y_2f(x)}{W(y_1,y_2)}dx$, $u_2=\int\frac{y_1f(x)}{W(y_1,y_2)}dx$
  $y_p(x)=y_1u_1+y_2u_2=y_1[-\int\frac{y_2f(x)}{W(y_1,y_2)}dx]+y_2[\int\frac{y_1f(x)}{W(y_1,y_2)}dx]$
  Wronskian: $W(y_1,y_2)=y_1y_2'-y_2y_1'\neq 0$
  
\end{mathline}

\chapter{Systems of Differential Equations}

\section{General Overview}

Systems of differential equations are systems of equations involving differentials. Listed here are three of many possible ways of solving a system.

\begin{itemize}
\item Substitution Method
\item Operator Method
\item Eigen Method
\end{itemize}

\section{Substitution Method}

Given:

$x'=ax+by+c$
$y'=dx+ey+f$

\begin{itemize}
\item Separate the variables in one of the equations.
\item Take the derivative of both sides.
\item Substitute into the other equation.
\item Solve the equation as a second order linear differential equation.
\end{itemize}

\section{Operator Method}

Given:

$x'=ax+by+c$
$y'=dx+ey+f$

\begin{itemize}
\item Let $D=\frac{d}{dt}$ and substitute into both equations
\item Bring all terms to one side
\item Factor out the dependent variables
\item Solve like a regular systems of equations
\end{itemize}

\section{Eigen Method}

Just like any systems of equations, a systems of differential equations can be represented in matrix form.

$\begin{pmatrix}
  x_0'\\
  x_1'\\
  .\\
  .\\
  .\\
  x_n'
\end{pmatrix}
=
\begin{pmatrix}
  a_0+a_1+...+a_n\\
  b_0+a_1+...+a_n\\
  ...\\
  ...\\
  ...\\
  ...
\end{pmatrix}
\begin{pmatrix}
  x_0\\
  x_1\\
  .\\
  .\\
  .\\
  x_n
\end{pmatrix}$
\newline{}

To find the characteristic equation of the system, find the kernel of the primary coefficient matrix.

$X_c(t):$
$det(A-\lambda)=0$
$\begin{vmatrix}
  A_{0,0}-\lambda & A_{0,1} \\
  A_{1,0} & A_{1,1}-\lambda \\
\end{vmatrix}=0$
solve for $\lambda$
for each $\lambda$, plug it back into the matrix to find the eigenvectors
$\begin{pmatrix}
  A_{0,0}-\lambda_0 & A_{0,1} \\
  A_{1,0} & A_{1,1}-\lambda_0 \\
  \end{pmatrix}\vec{v_0}=0$

$X_p(t):$
solve for the remaining terms as you would in the method of undetermined coefficients

\chapter{Laplace Transforms}
Definition:\newline{}
\laplace $\{f(t)\} =\int_0^\infty f(t)e^{-st}dt$
\section{Common Laplace Transform Forms}
Properties and Theorems of Laplace Transformations\newline{}
\begin{tabular}{|p{.35\linewidth}|p{.25\linewidth}|p{.4\linewidth}|}
  \hline
  Property/Theorem & t-space & S-space \\
  \hline
  Linearity & $af(t)+bf(t)$ & $aF(s)+bF(s)$ \\
  \hline
  Periodic Function & $f(t)=f(t+T)$ & $\frac{1}{1-e^{-Ts}}\int^T_0e^{-st}f(t)dt$\\
  \hline
  Unit Step Function & $u(t-\alpha)$ & $\frac{1}{s}e^{-\alpha s}$\\
  \hline
  Exponential Function & $e^{\alpha t}$ & $\frac{1}{s-\alpha}$ \\
  \hline
  Trig - cosine & $cos(\omega t)$ & $\frac{s}{s^2+\omega^2}$\\
  \hline
  Trig - sine & $sin(\omega t)$ & $\frac{\omega}{s^2+\omega^2}$\\
  \hline
  Polynomial Case 0  & $f(t)=1$ & $\frac{1}{s}$ \\ 
  \hline
  Polynomial Case 1 & $f(t)=t$ & $\frac{1}{s^2}$ \\
  \hline
  Polynomial Case n & $f(t)=t^n$ & $\frac{n!}{s^{n+1}}$ \\
  \hline  
  Deg. 1 Variable Coefficient & $tf(t)$ & $-F'(s)$\\
  \hline
  Deg. n Variable Coefficient & $t^nf(t)$ & $(-1)^nF^{n}(s)$\\
  \hline
  Derivative & $f'(t)$ & $sF(s)-f(0)$\\
  \hline
  Derivative & $f^{(n)}(t)$ & $s^nF(s)-\sum^n_{k=1}s^{n-k}f^{(k-1)}(0)$ \\
  \hline
  Convolution & $f(t)\star g(t)$ & $F(s)G(s)$\\
  \hline
\end{tabular}

\subsection{Convolution Theorem}
$\star$ refers to the convolution operator, a way of combining two functions into a third. The formal definition is as follows:
\begin{mathline}
  $f(t)\star g(t) = \int^t_0f(\tau)g(t-\tau)d\tau$
\end{mathline}

Convolution Theorem:
\begin{mathline}
\laplace $[f(t)\star g(t)]=F(s)G(s)$ and \laplace$^{-1}[F(s)G(s)]=f(t)\star g(t)$
\end{mathline}

Proof of the Convolution Theorem:\newline{}
\begin{mathline}
let $u(t)=1$ if $t\geq 0$ and $u(t)=0$ otherwise\newline{}
$[f(t)\star g(t)]u(t)$\newline{}
$(\int^{\inf}_0 f(\tau)g(t-\tau)d\tau) * u(t-\tau)$\newline{}
$\int^t_0 f(\tau)g(t-\tau)u(t-\tau)d\tau + \int^t_0 f(\tau)g(t-\tau)u(t-\tau)d\tau$\newline{}
$\int^t_0 f(\tau)g(t-\tau)d\tau$\newline{}
$f(t)\star g(t)= \int^{\inf}_0 f(\tau)g(t-\tau)u(t-\tau)d\tau$\newline{}
\laplace $[f(t)\star g(t)= \int^{\inf}_0 f(\tau)g(t-\tau)u(t-\tau)d\tau]$\newline{}
$\int^{\inf}_0\int^{\inf}_0 f(\tau)g(t-\tau)u(t-\tau)e^{-st}d\tau dt$\newline{}
$\int^{\inf}_0f(\tau)d\tau\int^{\inf}_0 g(t-\tau)u(t-\tau)e^{-st} dt$\newline{}
let $t=t_1+\tau$, $dt=dt_1$\newline{}
$\int^{\inf}_0f(\tau)d\tau\int^{\inf}_0 g(t_1)u(t_1)e^{-s(t_1+\tau)} dt_1$\newline{}
$\int^{\inf}_0f(\tau)e^{-s\tau}d\tau\int^{\inf}_0 g(t_1)u(t_1)e^{-st_1} dt_1$\newline{}
let $F(s)=\int^{\inf}_0f(\tau)e^{-s\tau}d\tau$, $G(s)=\int^{\inf}_0 g(t_1)u(t_1)e^{-st_1} dt_1$\newline{}
\laplace$[f(t)\star g(t)]=F(s)G(s)$
\end{mathline}

Properties of Convolutions
\begin{itemize}
\item commutative
\item associative
\item distributive
\end{itemize}

\section{Complex Inverses}
More complex inverses can be solved three ways:

\begin{itemize}
\item Separate an inverse using a convolution, solve the components separately, and plug the values into the convolution
\item Partial fraction decomposition into separate terms and solve individually
\end{itemize}

\end{document}

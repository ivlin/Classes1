\documentclass{article}

\usepackage{amsmath}
\usepackage{amssymb}
\usepackage{graphicx}

\begin{document}

Ivan Lin\newline{}
Dr. Esther Arkin\newline{}
AMS301\newline{}
1/25/17

\begin{center}
  Homework 1b
\end{center}

\underline{Section 1.2 Problem 6}

a. The graphs are isomorphic.\newline{}
$c=2$\newline{}
$a=5$\newline{}
$f=3$\newline{}
$b=6$\newline{}
$e=4$\newline{}
$d=1$\newline{}
The graphs are isomorphic. The graphas have the same number of vertices and edges (6 vertices, 11 edges). The degree pattern (5,4,4,3,3,3) are the same and with the given substitutions, adjacency is preserved.\newline{}\newline{}

b. The graphs are not isomorphic.\newline{}
The degree patterns are not the same. The graph on the laft has vertices of degrees (4,4,4,3,3,3) while the graph on the right has vertices of degrees (4,4,3,3,2,2).\newline{}

f. The graphs are not isomorphic.\newline{}
Adjacency is not preserved. In the graph on the left, all of the vertices with a degree of 4 are adjacent to two other degree-4 vertices, but in the graph on the right each of the degree-4 vertices are adjacent to only one other degree-4 vertex.\newline{}

h. The graphs are not isomorphic.\newline{}
Both graphs have the same number of vertices, but not the same amount of edges. The left graph has 15 edges while the right graph has 13 edges.\newline{}\newline{}

\underline{Section 1.2 Problem 8}

The first three graphs all have 8 vertices and 12 edges. Every vertex has a degree of 3, so adjacency is preserved. The fourth graph has 8 vertices but 14 edges while the fifth graph has 8 vertices but 13 edges.\newline{}
\end{document}

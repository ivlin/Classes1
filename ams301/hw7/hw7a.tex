\documentclass{article}

\usepackage{amsmath}
\usepackage{amssymb}
\usepackage{graphicx}

\begin{document}


Ivan Lin\newline{}
Dr. Esther Arkin\newline{}
AMS301\newline{}
3/19/17

\begin{center}
  Homework 7a
\end{center}

\underline{Section 5.1 Question 6}\newline{}
How many ways are there to pick a man and a woman who are not husband and wife from a group of \textit{n} married couples.\newline{}
There are $n(n-1)$ or $n^2-n$ ways of picking a man and woman who are not husband and wife. There are $n$ men to choose from for the male and $n-1$ women to choose from who are not his wife.\newline{}\newline{}

\underline{Section 5.1 Question 14}\newline{}
How many different numbers can be formed by various arrangements of the six digits $1, 1, 1, 1, 2, 3$?\newline{}
There are $\frac{6!}{4!}$ or $30$ different numbers that can be formed by arrangements of those six digits. There are $6$ digits to order, so that's $6!$ arrangments. However, there are 4 duplicates of $1$ and since we're only interested in different numbers, so we divide by $4!$ to account for arrangements that yield the same number due to the duplicates.\newline{}\newline{}

\underline{Section 5.1 Question 18}\newline{}
(a) How many different license plates involving three letters and two digits are
there if the three letters appear together either at the beginning or end of the
license?\newline{}
There are $26^3*2*10^2$ or $3515200$ different license plates that can be formed in such a way. There are $26^3$ ways of choose three ordered letters. However, they appear together at either the beginning or the end of the license plate, meaning that the amount must be multiplied by $2$ to represent the two possible locations the grouping of letters can appear. That is then multiplied by $10^2$ to account for the possible choices for the two digits.\newline{}\newline{}
(b) How many license plates involving one, two, or three letters and one,
two, or three digits are there if the letters must appear in a consecutive
grouping?\newline{}
\textit{Note: I do not understand exactly what the question is asking. I take it to mean that we must consider all possible combinations of letter and digit amounts (i.e. 1 letter + 1 digit, 1 letter + 2 digits, 2 letters + 1 digit, etc.)}\newline{}
If there are $[(26^1*2+10^1)+(26^2*2+10^1)+(26^3*2+10^1)]+[(26^1*3+10^2)+(26^2*3+10^2)+(26^3*3+10^2)]+[(26^1*4+10^3)+(26^2*4+10^3)+(26^3*4+10^3)]$ possible license plates. For a license plate with $m$ letters and $n$ numbers, there are $26^m$ possible letter permutations, multiplied by $n+1$ places the group of letters can occur among the digits, multiplied by $10^n$ permutations of digits, so there would be $26^m*(n+1)*10^n$ possible licenses. I found this term for all possible combinations of 1, 2, and 3 letters paired with 1, 2, and 3 digits.\newline{}\newline{}

\underline{Problem A}
Consider a decimal sequence of length 8 (decimal meaning digits 0,1,2,...,9 may appear). Each of
the following parts is independent of the others.\newline{}
(a). How many such sequences start and end with at least two 3’s?\newline{}
$10^4$ or $10000$ of the sequences start and end with at least two 3's. If at least the first two and last two digits must be 3, then the remaining 4 or fewer digits are free to be chosen from the digits from 0 to 9, so the answer is $10^4$.\newline{}\newline{}
(b). How many such sequences have exactly 2 different digits appearing (e.g. 05550005)?\newline{}
$\frac{10*9}{2*1}*(2^8-2)$ such sequences exist. A combination of two digits must be chosen to appear, and the order in which the two are selected do not matter, so $\frac{10*9}{2*1}$ accounts for the different possible digits. The $2^8-2$ term accounts for different arrangements the two digits in an 8-digit number. 2 is subtracted to account for the outcome where all of the numbers are the same, since exactly 2 digits must appear.\newline{}\newline{}
(c). How many such sequences have the digit 7 appearing at most 3 times?\newline{}
$9^5*\frac{8!}{5!3!}+9^6*\frac{8!}{6!2!}+9^7\frac{8!}{7!1!}+9^8\frac{8!}{8!1!}$ sequences have the digit 7 appearing 3 or fewer times. If the digit 7 occurs $n$ times in a sequence, we find the sum of the sequences where $n=0,1,2,3$. There are $8-n$ places that can be one of the nine digits besides 7, so the term includes $9^{8-n}$, and it is multiplied by the possible arrangements of those digits, $\frac{8!}{n!(8-n)!}$. The result is there are $9^{8-n}\frac{8!}{n!(8-n)!}$ sequences where the digit 7 occurs exactly $n$ times. The answer is the sum of terms where $n<3$.
\end{document}

\documentclass{article}

\usepackage{amsmath}
\usepackage{amssymb}
\usepackage{graphicx}

\begin{document}


Ivan Lin\newline{}
Dr. Esther Arkin\newline{}
AMS301\newline{}
4/22/17

\begin{center}
  Homework 10-b
\end{center}

\underline{Problem B}

Let $a_n$ be the number of sequences of blocks that can be placed consecutively in a line of length n if each block has length either 1 or 2. (Empty spots are not allowed.)\newline{}

(a). Write a recurrence for $a_n$. (You do not have to solve the recurrence.)\newline{}

$a_n=a_{n-1}+a_{n-2}$\newline{}


Reasoning: Since blocks are of size 1 or 2, a sequence of size $n$ can be formed by:
\begin{itemize}
  \item adding a size 1 block to a sequence of size $n-1$
  \item adding a size 2 block to a sequence of size $n-2$
  \item Note: adding 2 size 1 blocks would fall under the case of item 1
\end{itemize}

(b). Write a complete set of initial conditions.\newline{}

$a_1=1$ (1 size 1 block)\newline{}
$a_2=2$ (1 size 2 block or 2 size 1 blocks)\newline{}

(c). Calculate $a_6$ using your recursion from (a) and initial conditions from (b).\newline{}

$a_1=1$\newline{}
$a_2=2$\newline{}
$a_3=a_2+a_1=1+2=3$ \newline{}
$a_4=a_3+a_2=3+2=5$ \newline{}
$a_5=a_4+a_3=5+3=8$\newline{}
$a_6=a_5+a_4=8+5=13$\newline{}

(d). Now assume that no 2 blocks of length 1 can be placed consecutively. Write a recurrence for $a_n$. (You do not have to solve the recurrence.)\newline{}

$a_n=a_{n-2}+a_{n-3}$\newline{}

Reasoning:\newline{}

A sequence can be made to length $n$ by adding a block of size of 2 to a sequence of size $n-2$ or adding a block of size 1 to a sequence of size $n-1$ if it ends in a size 2 block. The latter case is only possible by adding a size 2 block to a sequence of length $n-3$.\newline{}

(e). Write a complete set of initial conditions for your recurrence of part (d).\newline{}

$a_1=1$ (1 size 1 block)\newline{}
$a_2=1$ (1 size 2 block)\newline{}
$a_3=2$ (size 1 block + size 2 block, size 2 block + size 1 block)\newline{}

\underline{Problem C}

Solve the following recurrence relation: $a_n=4a_{\frac{n}{2}}+3n$, $a_1=1$\newline{}
(You may assume that $n=2^m$, for some $m=0,1,2,...$)

$a_n=4a_{\frac{n}{2}}+3n$, $a_1=1$\newline{}
$c=4$, $c\neq k$, $f(n)=3n$\newline{}
when $c\neq k, f(n)=dn, a_n=An^{log_kc}+(\frac{kd}{k-c})n$\newline{}
$a_n=n^{log_24}+\frac{2*3}{2-4}n$\newline{}
$a_n=n^2-3n$\newline{}

\underline{Problem D}

Consider the recurrence relation: $a_n=2a_{n-1}+1, a_1=1$\newline{}
Prove that the solution to this recurrence is $a_n=2^{n}-1$\newline{}

proof by induction\newline{}
base case: $a_1=1$\newline{}
induction hypothesis: we will prove for all $n$ that $a_n=2a_{n-1}+1=a_n=2^n-1$\newline{}
induction step: we will assume the relation is true for $n-1$\newline{}
$a_{n-1}=2a_{n-2}+1=2^{n-1}-1$ \newline{}
$2a_{n-1}=4a_{n-2}+2=2^{n}-2$ - multiply by 2\newline{}
$2a_{n-1}=2(a_{n-2}+1)=2^n-2$ - factor out\newline{}
$2a_{n-1}=2a_{n-1}=2^n-2$ - substitute\newline{}
$2a_{n-1}=2a_{n-1}+1=2^n-1$ - add 1\newline{}
we have proven the base case and the induction step true, therefore by the law of induction, we have proven the proposition true
\end{document}

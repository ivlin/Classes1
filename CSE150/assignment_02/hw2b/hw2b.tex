\documentclass{article}

\usepackage[english]{babel}
\usepackage[utf8x]{inputenc}
\usepackage{amsmath}
\usepackage{amssymb}
\begin{document}

Ivan Lin

Dr. Michael Bender

CSE150 - Honors Foundations of Computer Science Fall 2016

\begin{center}
Homework 2b
\end{center}

\underline{Problem 5}

Show that each function $f: \mathbb{N} \implies \mathbb{N}$ has the listed properties.

1. $f(x)=2x$ (one-to-one but not onto)

$f'(x)=\frac{x}{2}$

There exists only one $y$ for any $x$, thus it is injective. There exists $f(x)$ in $\mathbb{N}$ that do not have a corresponding $x$ in $\mathbb{N}$, when $f(x)$ is odd and does not divide evenly into $2$. This means the function is not surjective.\newline


2. $f(x)=x+1$ (one-to-one but not onto)

$f'(x)=x-1$

There exists only one $y$ for any $x$, thus it is injective. However, depending on the definition of $\mathbb{N}$ and whether it includes $0$, the smallest element of the domain has no corresponding $y$ value in $\mathbb{N}$\newline


3. $f(x)=$ if $x$ is odd then $x-1$ else $x+1$ (bijective)

When $x$ is odd, $f(x)=x-1$ is even, $x=\{1,3,5,...\},f(x)=\{0,2,4,..\}$, and when $x$ is even, $f(x)=x+1$ is odd $x=\{0,2,4,...\},f(x)=\{1,3,5,...\}$. The domain and range cover all $\mathbb{N}$. The function is bijective.\newline


\underline{Problem 6}

Show that the product $(a+bi)(c+di)$ of two complex numbers can be evaluated using just three real number multiplications. You may use a few extra additions.

$(a+bi)(c+di)$\newline
$ac+adi+cbi-bd$\newline
$ac-bd+adi+cbi$

At this point, I tried several different methods, but failed to get the answer. I then looked up the answer and subsequently put my palm to my face in an act of frustration.

$ac-bd+i[(a-b)(d-c)+ac+bd]$\newline


\underline{Problem 7}

Given a function $f: A\implies A$. An element $a \in A$ is called a fixed point of $f$ if $f(a)=a$. Find the set of fixed points for each of the following functions.

1. $f: A \implies A$ where $f(x)=x$.

$\forall a \in \mathbb{R} \implies a \in A$\newline

2. $f:\mathbb{N}\implies \mathbb{N}$ where $f(x)=x+1$

$\varnothing$\newline

3. $f:\mathbb{N}_6 \implies \mathbb{N}_6$ where $f(x)=2x$ mod 6.

$\{0\}$\newline

4. $f:\mathbb{N}_6 \implies \mathbb{N}_6$ where $f(x)=3x$ mod 6.

$\{0,3\}$\newline

\underline{Problem 8}
Let $f(x)=x^2$ and $g(x,y)=x+y$. Find compositions that use the functions $f$ and $g$ for each of the following expressions.\newline

1. $(x+y)^2$

$f(g(x,y))$\newline

2. $x^2+y^2$

$g(f(x),f(y))$\newline

3. $(x+y+z)^2$

$f(g(x,y),z)$\newline

4. $x^2 + y^2 + z^2$

$g(g(f(x),f(y)),f(z))$

\end{document}

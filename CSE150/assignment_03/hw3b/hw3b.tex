\documentclass[12pt]{article}[12pt]
\usepackage[left=0.5in,right=0.5in,top=0.5in,bottom=0.5in]{geometry}
\usepackage[english]{babel}
\usepackage[utf8x]{inputenc}
\usepackage{amsmath}
\usepackage{amssymb}
\usepackage{scrextend}

\begin{document}

Ivan Lin

Dr. Michael Bender

CSE150 - Honors Foundations of Computer Science Fall 2016

\begin{center}
Homework 3b
\end{center}

\underline{Problem 9}

Let $V$ represent the volume of the pickle, $V_w$ represent the volume of water in the pickle, $e$ represent the fraction of water the has evaporated.

The problem become a system of equations.


\begin{addmargin}[1em]{2em}

\indent$.99V=V_w$

\indent$.98(V-eV_w)=V_w-eV_w$

\indent Solve equation 1 for $V$ and substitute into equation 2.

\indent By equation 1, $V=\frac{V_w}{.99}$

\indent$.98(\frac{V_w}{.99}-eV_w)=V_w-eV_w$

\indent Distribute and get the $e$ terms on the same side

\indent$\frac{98}{99}V_w-.98eV_w=V_w-eV_w$

\indent$0.02eV_w=\frac{1}{50}eV_w=\frac{1}{99}V_w$

\indent Simplify

\indent$eV_w=\frac{50}{99}V_w$

\indent$e=\frac{50}{99}$

This means $\frac{50}{99}$ of the water evaporated, leaving $\frac{49}{99}$ of the water remaining. This means the new mass of the water is $mass*(1-e)=10000*\frac{49}{99}=\frac{490000}{99}$. So $\frac{490000}{99}$ kilograms of water remains.
\end{addmargin}
\bigskip
\underline{Problem 10}
\smallskip
Prove the following using mathematical induction:

1. $2n \leq 2^n$

\begin{addmargin}[1em]{2em}

Proof: proof by induction on $n$

Let $P(n)$ be the predicate $2n\leq 2^n$

Base Case: P(0) is true because:

$2(0)\leq 2^0$

$0\leq 1$

Induction Step: Assume $n$ is such that $2n\leq 2^n$

$2n\leq 2^n$

$2 * 2n \leq 2^n*2$

$4n \leq 2^{n+1}$

$2n+2\leq 2^n+2$

$2(n+1)\leq 2^n+2$

since $n\geq 1$ for all $n\in \mathbb{N}$

$2n\geq 2$ for all $n\in \mathbb{N}$

$4n\geq 2n+2$ for all $n\in \mathbb{N}$

$2(n+1)\leq 4n \leq 2^{n+1}$

Therefore $P(n+1)$ holds true given $P(n)$ is true.

Since the base case is true and the induction step is true, $P(n)$ holds for all $n$ by the principle of mathematical induction.
\end{addmargin}
\bigskip
2. $1+3+5+...+(2n-1)=n^2$

\begin{addmargin}[1em]{2em}

Proof: proof by induction on $n$

Let $P(n)$ be the predicate $1+3+5+...+(2n-1)=n^2$

Base Case: $P(0)$ is true because $0=0^2$

Induction Step: Assume $P(n)$ hold for $n$ so $1+3+5+...+(2n-1)=n^2$

$1+3+5+...+(2n-1)=n^2$

$1+3+5+...+(2n-1) + 2n+1 =n^2 + 2n+1$

$1+3+5+...+(2n-1) + (2n + 1) = (n+1)(n+1)$

$1+3+5+...+(2n-1) + (2n + 1) = (n+1)^2$

Therefore $P(n+1)$ holds true.

Since the base case is true and the induction step is true, the predicate $P(n)$ is true by the principle of mathematical induction.

\bigskip

3. $1^2+2^2+3^2+...+n^2=\frac{(n)(n+1)(2n+1)}{6}$

Proof: proof by induction of $n$

Let $P(n)$ be the predicate $1^2+2^2+3^2+...+n^2=\frac{(n)(n+1)(2n+1)}{6}$

Base Case: $P(0)$ is true because $0=\frac{(0)(1)(1)}{6}$

Induction Step: Assume $P(n)$ holds for $n$ so $1^2+2^2+3^2+...+n^2=\frac{(n)(n+1)(2n+1)}{6}$

$1^2+2^2+3^2+...+n^2 + (n+1)^2=\frac{(n)(n+1)(2n+1)}{6}+(n+1)^2$

$1^2+2^2+3^2+...+n^2 + (n+1)^2=\frac{(n)(n+1)(2n+1)}{6}+\frac{6(n+1)^2}{6}$

$1^2+2^2+3^2+...+n^2 + (n+1)^2=\frac{(n+1)[(n)(2n+1)+6(n+1)]}{6}$

$1^2+2^2+3^2+...+n^2 + (n+1)^2=\frac{(n+1)[2n^2+n+6n+6)]}{6}$

$1^2+2^2+3^2+...+n^2 + (n+1)^2=\frac{(n+1)[2n^2+7n+6)]}{6}$

$1^2+2^2+3^2+...+n^2 + (n+1)^2=\frac{(n+1)(n+2)(2n+3)]}{6}$

$1^2+2^2+3^2+...+n^2 + (n+1)^2=\frac{(n+1)(n+2)(2n+3)]}{6}$

$1^2+2^2+3^2+...+n^2 + (n+1)^2=\frac{(n+1)(n+2)(2(n+1)+1)]}{6}$

Therefore $P(n+1)$ holds true.

Since the base case is true and the induction step is true, the predicate $P(n)$ is true by the principle of mathematical induction.
\end{addmargin}
\bigskip
\underline{Problem 11}
You have an $n \times m$ bar of chocolate. Your goal is to separate all of the squares of chocolate. The way that you can break the chocolate is to take a single piece of chocolate (connected component of squares) and break it along one horizontal or vertical line. What is the minimum number of breaks necessary? Please prove your answer.

\begin{addmargin}[1em]{2em}

It takes $n*m-1$ breaks to create $n*m$ pieces of chocolate

Proof: 

proof by induction on $k=n*m$, or the number of pieces of chocolate

Let $P(k)$ be the predicate that it takes $k-1$ breaks to create $k$ pieces of chocolate

Base Case: It takes 0 breaks to break one piece of chocolate into 1 piece. $0=k-1=1-1$

Induction Step: Assume $P(k)$ holds for $k$, meaning it takes $k-1$ steps to create $k$ pieces of chocolate. One more break on one of those pieces of chocolate will create a one new piece of chocolate, so $k$ breaks will create $k+1$ pieces of chocolate. $P(k+1)$ is true.

Since the base case is true and the induction step is true, the predicate $P(k)$ is true by the principle of mathematical induction.
\end{addmargin}
\bigskip
\underline{Problem 12}
You have an $n×n$ checkerboard with an initial set of checkers placed on it. You are allowed to add additional checkers under the following conditions: You can place a checker on a square if two or more neighboring squares also have checkers on them. Neighboring cells are those above, below, to the left and to the right, as shown in Figure 2. As we showed in class, there are initial configurations of n checkers that enable the entire board to be covered. Prove that no configuration of $n − 1$ checkers can let you cover the board.

\begin{addmargin}[1em]{2em}

Proof:

we will first prove that the perimeter of the covered region of a checkerboard with any amount of checkers is monotonically decreasing 

proof by induction on $t$, the amount of time that passes

Let $P(t)$ be the predicate that at time $t$, the perimeter has not increased 

Base Case: $P(0)$ is true because at time 0, the perimeter will not have increased 

Induction Step: Assume $P(t)$ is true at time $t$ and the perimeter has not increased

There are two possible ways a space on the board can be filled in and the size of the covered region will increase. Given two checkers, the perimeter of the covered regions will add up to $8$ at most, and less if they are adjacent to each other. If they fall along the same column or row and have a space between them, that space will fill. The size of the new checker configuration will be 3 squares, with a perimeter of $8$. In the case two checkers are diagonally adjacent, only one or both of the two squares adjacent to both filled spaces can be filled and the new cofiguration of either 3 or 4 checkers will have a perimeter of $8$. If no new checker is added at time $t+1$, 

We have proven $P(t+1)$ true.

Since we have proven the base case and the induction step true, the predicate $P(t)$ is true by the principle of mathematical induction.

For any cofiguration of covered checkers, the perimeter of the covered regions will never increase to a number greater than the perimeter of the original region

Since the maximum perimeter of a board containing $n-1$ checkers is $4(n-1)$ or $4n-4$, and the perimeter of a covered board is $4n$, this lemma shows that a board with $n-1$ checkers will never be completely covered.
\end{addmargin}
\end{document}
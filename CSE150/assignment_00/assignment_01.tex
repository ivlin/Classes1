\documentclass[a4paper]{article}

\usepackage[english]{babel}
\usepackage[utf8x]{inputenc}

\begin{document}

Ivan Lin

Dr. Michael Bender

CSE150 - Honors Foundations of Computer Science Fall 2016

Academic Honesty Review
\newline
\newcounter{questions}
\stepcounter{questions}


\underline{Question \arabic{questions}}
Explain why we let students work together to solve problems, as long as the students cite their collaborators. Explain why working together is not academically dishonest in this course.

Students are allowed to work in groups because it promotes teamwork, which is an essential skill in many disciplines, including computer science. Of course, students are expected to learn to properly cite their collaborators and make their own contributions to assignments, not just copy another's answers verbatim. 


\stepcounter{questions}
\underline{Question \arabic{questions}}
Explain why it is important to your professional development to struggle with a problem that you cannot solve quickly. In other words, the instructor deliberately assigns homework he knows you will likely have to think about for days or weeks to solve. What do you expect to learn from this
experience?

It is important that students learn to struggle with problems they can't quickly solve, since it shows persistence and determination on the behalf of the student. Furthermore, working on a problem teaches problem solving skills. Students will have to face a difficult problem, take breaks if they can't solve it, and reapproach the problem another day with fresher eyes. Especially for those going into research, students will eventually face problems that requires work, time and commitment. 


\stepcounter{questions}
\underline{Question \arabic{questions}}
Explain why, although it is ok to work with other students, it is plagiarism to share and/or copy other write-ups. Give an example of collaboration that is academically honest. Give another example of collaboration that is academically dishonest

Plagiarism is the act of passing another's words or ideas as ones own. So long as one makes their own contribution when collaborating with others and cites the other collaborators, a student is absolutely allowed to submit something multiple people have shared the work on. For example, if two students discuss possible solutions to a complex problem and mention the other in their answer to the assignment, that would be honest collaboration. An example of academic dishonesty would be if one student gives another a look at their assignment and the other simply copies the answer word for word.

\stepcounter{questions}
\underline{Question \arabic{questions}}
Explain why it is academically dishonest to share your solution set with another student. Explain how you could get burned from just sharing your writeup even if you do not copy yourself.

Sharing work is academic dishonesty because you know the student will be academic dishonesty and you are complicit in helping the copier trick the teacher. FUrthermore, it is disrespectful to the other students who struggled and honestly worked on the problems.

\stepcounter{questions}
\underline{Question \arabic{questions}}
Explain why copying (or approximately copying) solutions from the web (or another source) is plagiarism, even if you cite your source.

If a problem is your own work, you should have made your own contribution to it. Even if you tried prior to simply copying and citing a web source, you will not have showed any understanding of the problem or solution if you did not add your own thoughts to the problem.

\stepcounter{questions}
\underline{Question \arabic{questions}}
Explain why it is better for your grade to leave a question blank, rather than search for answers on the web. (Hint: calculate approximately how much a homework problem is worth to your raw score versus an exam question. Feel free to include the risk-benefit analysis of getting caught.)

While the a single problem on the homework portion of one's grade is insignificant, a single problem on an exam is worth far more. If one copies a web problem, they likely do not understand it, have not asked for someone to explain it, and the professor would not cover it in class if they see the student answered it correctly. The student would then get the problem incorrect if it showed up on an exam, especially if there are superficial changes that make the problem appear different. And of course, there's always the risk of suspension if one gets caught.

\stepcounter{questions}
\underline{Question \arabic{questions}}
Imagine that you are employed at a major software company, say Google, Facebook, or Microsoft, and commit code into a product that you copied from a website. Explain the potential risks to both you and the company if this action is discovered by the owners of the code.

Committing copied code is incredibly dangerous since it opens yourself and the company up to lawsuits and could completely invalidate the product, forcing the company to have to rewrite it. The result is costly in terms of money and time. The worst case scenario could even be jailtime for using proprietary code. Last of all, if you copy code you don't understand, it could potentially cause bugs in the product.

\stepcounter{questions}
\underline{Question \arabic{questions}}
Please speculate on why we decided to make a problem set on academic honesty.

If students answer a problem set on academic dishonesty, it means they can't claim ignorance of the rules. They must acknowledge what counts as dishonesty and what is permissible. There is no ambiguity regarding what they can or can't do.

\stepcounter{questions}
\underline{Question \arabic{questions}}
How much time did this writeup take you, including the time it took to learn latex.

The writeup took me roughly an hour, including looking up the latex markup. The time it took to "learn" latex is more complicated, since I already had some familiarity with latex though I have never worked with it myself. In addition, I only know the basics, and will probably be working with a reference sheet when I use latex in the future.

\end{document}

\documentclass{article}

\usepackage[english]{babel}
\usepackage[utf8x]{inputenc}
\usepackage{amsmath}
\begin{document}

\textbf{Solution developed in collaboration with Andy Liang}

\underline{Here is an explanation written up by me}\newline

Given $n$ professors and a huge number $L$.\newline

Each professor randomly chooses a random number $R$ uniformly distributed from 0 to $L$.\newline

Each professor then randomly distributes $R$ into $n$ components (divided randomly, not evenly) that add up to $N$, where each component is represented by $r$.\newline

Every professor receives one of these components, and this is true for each professor, so each professor eventually has their salary, $S$, and the sum of the salary components from other professors, $\sum_{i=1}^{n} r_i$.\newline

The professors each take $S+\sum_{i=1}^{n} r_i$ (the sum of their salary and the sum of the components they received) together. \newline

Each takes that total modulo $L$ (which obfuscates the expected value), and add them all together with that of other professors.\newline

$\sum_{i=1}^{n} [(S_i + \sum_{j=1}^{n} r_{i,j})\%L] = \sum_{i=1}^{n} [(S_i + R_i)\%L$].\newline

The total is equal to the sum of all salaries and all individual components, which is also equal to the sum of all salaries and all the originally chosen random numbers.\newline

Each professor than subtracts their original random number, $R_i$, from the total.\newline

\noindent\rule{10cm}{0.4pt}\newline

\textbf{Note} Alternatively, each professor add $(S + R)\%L$ (their salary and their original random number) to the total. Each professor then subtracts $\sum_{i=1}^{n} r_i$ (the sum of the components they receive) from the sum total.\newline

\noindent\rule{10cm}{0.4pt}\newline

Regardless of the method, $\sum_{i=1}^n R_i = \sum_{i=1}^n \sum_{j=1}^n r_{i,j}$. The total sum of the random numbers are equal to that of the random components. However, each individual professor's random number $R$ and component sum $\sum_{i=1}^n r_i$ are different, so it is impossible to discern each professor's individual salary.\newline

Once the random numbers have been subtracted and the difference has been found, the answer should first be taken modulo $L$. The output is then $\sum_{i=1}^n S_i$, which can be divided by $n$ to find the average salary of the professors.

\newpage
\underline{Here is an explanation written up by Andy (for documentation purposes)}

\noindent
\textbf{Collaborated with Ivan Lin}
\\* Similar to part a, the professors start by agreeing on a number L that has a value much larger than one trillion. The first professor picks a random number R$_1$ between 0 and L-1 inclusive. However, instead of the first professor adding his salary to the random number, he splits it into n smaller random numbers that add up to the initial random number r$_1$ to r$_n$. He then randomly picks a number between r$_1$ to r$_n$ inclusive before distributing the rest to the other professors. After the other n-1 professors repeat the same process, each professor should have two random numbers: the random number they chose between 0 and L -1 (R$_n$) and the sum of all the random pieces they were given ($\sum_{i=1}^{n} r_{i}$). 
The first professor then adds either R$_1$ or $\sum_{i=1}^{n} r_{i}$ to his salary (for the sake of this explanation, I'll say he uses R$_1$) and mods it by L to maintain a uniform distribution. 
\smallskip
\\*\centerline{Current Total = (S$_1$ + R$_1$) mod L}
\smallskip 
\\*Rather than telling the professor next to him, the first professor says his number aloud for all professors to hear. The next professor volunteers and then adds his salary (S$_2$) plus his random number (R$_2$) modded by L to the current total.
\\*\centerline{Current Total = (S$_1$ + R$_1$ + S$_2$ + R$_2$) mod L}
\smallskip
\\*The remaining professors do the same resulting in:
\smallskip
\\*\centerline{Total = (Total Professor Salaries ($\sum_{i=1}^{n} S_{i}$) + Total Random Numbers ($\sum_{i=1}^{n} R_{i}$)) mod L}
\smallskip
\\*To figure out the total professor salaries, the professors all subtract the sum of their random pieces ($\sum_{i=1}^{n} r_{i}$) then mod by L
\\*\centerline{$\sum_{i=1}^{n} R_{i}$ = $\sum_{i=1}^{n}\sum_{i=1}^{n} r_{i}$}
\smallskip
\\*\centerline{Total Professor Salaries ($\sum_{i=1}^{n} S_{i}$) = (Total - Total Random Pieces ($\sum_{i=1}^{n}\sum_{i=1}^{n} r_{i}$)) mod L}
\smallskip
\\*To find the average of their salaries, the professors divide their total by n.
\smallskip
\\*\centerline{Average Professor Salary = Total Professor Salaries / n}


\end{document}
